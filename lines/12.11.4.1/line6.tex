\documentclass[12pt]{article}
\usepackage{graphicx}
\usepackage{amsmath}
\usepackage{mathtools}
\usepackage{gensymb}
\usepackage[utf8]{inputenc}
\usepackage{float}
\newcommand{\mydet}[1]{\ensuremath{\begin{vmatrix}#1\end{vmatrix}}}
\providecommand{\brak}[1]{\ensuremath{\left(#1\right)}}
\providecommand{\norm}[1]{\left\lVert#1\right\rVert}
\newcommand{\solution}{\noindent \textbf{Solution: }}
\newcommand{\myvec}[1]{\ensuremath{\begin{pmatrix}#1\end{pmatrix}}}
\let\vec\mathbf

\begin{document}
\begin{center}
\textbf\large{CLASS-12 \\ CHAPTER-11 \\ THREE DIMENSIONAL GEOMETRY}
\end{center}
\section*{Excercise 11.4}

Q1. Show that the line joining the origin to the point $(2, 1, 1)$ is perpendicular to the
line determined by the points $(3, 5, – 1), (4, 3, – 1)$.
\\
\solution
\\
Given:
\begin{align}
  \vec{P}=\myvec{2\\1\\1},\vec{A}=\myvec{3\\5\\-1},\vec{B}=\myvec{4\\3\\-1}
\end{align}
From the given points $\vec{A}\text{ and }\vec{B}$, the directional vector is given by:
		\begin{align}
	\vec{m}=\vec{A}-\vec{B}=\myvec{-1\\2\\0}
		\end{align}
			 check whether two vectors are perpendicular to each other or not using:
		\begin{align}
			\vec{m}^\top\vec{P}=&0\\
			\myvec{-1&2&0}\myvec{2\\1\\1}&=0
		\end{align}
				Hence, it satisfies the above condition, shows that the line passing through origin is perpendiclar to the line passing through points $\vec{A} \text{ and } \vec{B}$.
		
\end{document}

