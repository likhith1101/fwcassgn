\documentclass[12pt]{article}
\usepackage{graphicx}
\usepackage{amsmath}
\usepackage{mathtools}
\usepackage{gensymb}
\usepackage[utf8]{inputenc}
\usepackage{float}
\newcommand{\mydet}[1]{\ensuremath{\begin{vmatrix}#1\end{vmatrix}}}
\providecommand{\brak}[1]{\ensuremath{\left(#1\right)}}
\providecommand{\norm}[1]{\left\lVert#1\right\rVert}
\newcommand{\solution}{\noindent \textbf{Solution: }}
\newcommand{\myvec}[1]{\ensuremath{\begin{pmatrix}#1\end{pmatrix}}}
\let\vec\mathbf

\begin{document}
\begin{center}
\textbf\large{CLASS-12 \\ CHAPTER-11 \\ THREE DIMENSIONAL GEOMETRY}
\end{center}
\section*{Excercise 11.2}

Q1. Show that the three lines with direction cosines $\frac{12}{13}, \frac{-3}{13}, \frac{-4}{13}; \frac{4}{13}, \frac{12}{13}, \frac{3}{13}; \frac{3}{13}, \frac{-4}{13}, \frac{12}{13}$ are mutually perpendicular.
\\
\solution
\begin{enumerate}
\item Check whether angle between the $\vec{A} \text{ and } \vec{B}$ are mutually perpendicular or not:
		\begin{align}
		\vec{A}=\myvec{\frac{12}{13}\\\\\frac{-3}{13}\\\\\frac{-4}{13}\\\\},\vec{B}=\myvec{\frac{4}{13}\\\\\frac{12}{13}\\\\\frac{3}{13}\\\\},\vec{C}=\myvec{\frac{3}{13}\\\\\frac{-4}{13}\\\\\frac{12}{13}\\\\}\\
	\vec{P}=\myvec{\frac{12}{13}&\frac{4}{13}    &\frac{3}{13}\\\\ \frac{-3}{13}&\frac{12}{13}&\frac{-4}{13}\\\\ \frac{-4}{13}&\frac{3}{13}&\frac{12}{13}\\\\},\vec{P}^\top=\myvec{\frac{12}{13}&\frac{-3}{13}   &\frac{-4}{13}\\\\ \frac{4}{13}&\frac{12}{13}&\frac{3}{13}\\\\ \frac{3}{13}&\frac{-4}{13}&\frac{12}{13}\\\\}
		\end{align}
	Check whether all three vectors are orthogonal to each other or not using:
		\begin{align}
		\vec{P}.\vec{P}^\top=\vec{I}
		\end{align}
		\begin{align}
	\myvec{\frac{12}{13}&\frac{4}{13}    &\frac{3}{13}\\\\ \frac{-3}{13}&\frac{12}{13}&\frac{-4}{13}\\\\ \frac{-4}{13}&\frac{3}{13}&\frac{12}{13}\\\\}.\myvec{\frac{12}{13}&\frac{-3}{13}   &\frac{-4}{13}\\\\ \frac{4}{13}&\frac{12}{13}&\frac{3}{13}\\\\ \frac{3}{13}&\frac{-4}{13}&\frac{12}{13}\\\\}=\myvec{1&0&0\\0&1&0\\0&0&1}
		\end{align}
		Hence, all three lines are perpendicular to each other.
	Angle between the vectors is given by:
		\begin{align}
			\cos\theta_1&=\frac{\vec{A}^\top\vec{B}}{\norm{\vec{A}}\norm{\vec{B}}}\\
			&=\frac{\myvec{\frac{12}{13}&\frac{-3}{13}&\frac{-4}{13}}\myvec{\frac{4}{13}\\\\\frac{-3}{13}\\\\\frac{-4}{13}\\}}{1}\\
			&=0\\
			\implies \theta_1&=90\degree
		\end{align}
	\item Check whether angle between the $\vec{B} \text{ and } \vec{C}$ are mutually perpendicular or not:
                  \begin{align}
  \vec{B}=\myvec{\frac{4}{13}\\\\\frac{-3}{13}\\\\\frac{-4}{13}\\},\vec{C}=\myvec{\frac{3}{13}\\\\\frac{-4}{13}\\\\\frac{-12}{13}\\}
		  \end{align}
Angle between the vectors is given by:
\begin{align}
                          \cos\theta_2&=\frac{\vec{B}^\top\vec{C}}{\norm{\vec{B}}\norm{\vec{C}}}\\
        &=\frac{\myvec{\frac{4}{13}&\frac{-3}{13}&\frac{-4}{13}}\myvec{\frac{3}{13}\\\\\frac{-4}{13}\\\\\frac{-12}{13}\\}}{1}\\
                          &=0\\
		  \implies \theta_2&=90\degree
		  \end{align}
	  \item Check whether angle between the $\vec{A} \text{ and } \vec{C}$ are mutually perpendicular or not:
                   \begin{align}
    \vec{A}&=\myvec{\frac{12}{13}\\\\\frac{-3}{13}\\\\\frac{-4}{13}\\}, \vec{C}=\myvec{\frac{3}{13}\\\\\frac{-4}{13}\\\\\frac{-12}{13}\\}    
		   \end{align}
	Angle between the vectors is given by:
	\begin{align}
                          \cos\theta_3&=\frac{\vec{A}^\top\vec{C}}{\norm{\vec{A}}\norm{\vec{C}}}\\
   &=\frac{\myvec{\frac{12}{13}&\frac{-3}{13}&\frac{-4}{13}}\myvec{\frac{3}{13}\\\\\frac{-4}{13}\\\\\frac{-12}{13}\\}}{1}\\                  
			   &=0\\
		   \implies \theta_3&=90\degree  
		   \end{align}
Hence, all three lines are mutually perpendicualr to each other.
\end{enumerate}
\end{document}

