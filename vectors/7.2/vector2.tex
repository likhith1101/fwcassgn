\documentclass[12pt]{article}
\usepackage{graphicx}
\usepackage{amsmath}
\usepackage{mathtools}
\usepackage{gensymb}

\newcommand{\mydet}[1]{\ensuremath{\begin{vmatrix}#1\end{vmatrix}}}
\providecommand{\brak}[1]{\ensuremath{\left(#1\right)}}
\providecommand{\norm}[1]{\left\lVert#1\right\rVert}
\newcommand{\solution}{\noindent \textbf{Solution: }}
\newcommand{\myvec}[1]{\ensuremath{\begin{pmatrix}#1\end{pmatrix}}}
\let\vec\mathbf

\begin{document}
\begin{center}
\textbf\large{CHAPTER-7 \\ COORDINATE GEOMETRY}
\end{center}
\section*{Excercise 7.2}

1. Find the coordinates of the point which divides the join $\brak{-1,7} and \brak{4,-3}$ in the ratio 2:3 :
\\
\\
\solution

 The coordinates are given as
\begin{align}
\vec{P} = \myvec{
-1\\
7\\
},
\vec{Q} = \myvec{
4\\
-3\\
},  n=\frac{3}{2}
\end{align}
\begin{align}
\vec{R} =\frac{\vec{Q}+n\vec{P}}{1+n}
\end{align}


\begin{align}
\vec{R}=\frac{\myvec{
4\\
-3\\
}
  +
   \frac{3}{2}\myvec{
-1\\
7\\
}}{1+\frac{3}{2}}
\end{align}

\begin{align}
\vec{R}=\frac{\myvec{
4\\
-3\\
}
  +
  \myvec{
-5\\
7\\
}}{1+\frac{3}{2}}
\end{align}

\begin{align}
\vec{R}=\frac{\myvec{
4\\
-3\\
}
  +
  \frac{1}{2}\myvec{
-3\\
21\\
}}{1+\frac{3}{2}}
\end{align}
 
\begin{align}
\vec{R}=\frac{\frac{1}{2}\myvec{
5\\
15\\
}}
  {\frac{5}{2}}
\end{align}

\begin{align}
\vec{R}=\frac{1}{2} . \frac{2}{5}\myvec{
5\\
15\\
}
\end{align}

\begin{align}
\vec{R}=\frac{1}{5} \myvec{
5\\
15\\
}
\end{align}


\begin{align}
\vec{R}=\myvec{
1\\
3\\
}
\end{align}





Hence, the coordinates of the point which divides the join is $R\brak{1,3}$ also shown in Figure:\ref{fig:Fig}


\begin{figure}[!h]
\begin{center}
   \includegraphics[width=\columnwidth]{./figs/linefig.png}
\end{center}
\caption{}
\label{fig:Fig}
\end{figure}

\end{document}

