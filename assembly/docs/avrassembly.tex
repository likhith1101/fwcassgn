% #######################################
% ########### FILL THESE IN #############
% #######################################
\def\mytitle{ BOOLEAN LOGIC IMPLEMENTATION  BY USING   ARDUINO WITH AVR ASSEMBLY}
\def\mykeywords{}
\def\myauthor{GADDAM LIKHITHESHWAR}
\def\contact{glikhitheshwar@gmail.com}
\def\mymodule{}
% #######################################
% #### YOU DON'T NEED TO TOUCH BELOW ####
% #######################################
\documentclass[12pt, a4paper]{article}
\usepackage[a4paper,outer=1.5cm,inner=1.5cm,top=1.75cm,bottom=1.5cm]{geometry}
\twocolumn
\usepackage{graphicx}
\graphicspath{{./images/}}
\usepackage{karnaugh-map}
%colour our links, remove weird boxes
\usepackage[colorlinks,linkcolor={black},citecolor={blue!80!black},urlcolor={blue!80!black}]{hyperref}
%Stop indentation on new paragraphs
\usepackage[parfill]{parskip}
%% Arial-like font
\usepackage{lmodern}
\renewcommand*\familydefault{\sfdefault}
%Napier logo top right
\usepackage{watermark}
%Lorem Ipusm dolor please don't leave any in you final report ;)
\usepackage{circuitikz}
\usetikzlibrary{calc}
\usepackage{tikz}

\usetikzlibrary{shapes, arrows, chains, decorations.markings,intersections,calc}
\usepackage{lipsum}
\usepackage{xcolor}
\usepackage{listings}
%give us the Capital H that we all know and love
\usepackage{float}
%tone down the line spacing after section titles
\usepackage{titlesec}
%Cool maths printing
\usepackage{amsmath}
\usepackage{tabularx}
%PseudoCode
\usepackage{algorithm2e}

\titlespacing{\subsection}{0pt}{\parskip}{-3pt}
\titlespacing{\subsubsection}{0pt}{\parskip}{-\parskip}
\titlespacing{\paragraph}{0pt}{\parskip}{\parskip}
\newcommand{\figuremacro}[5]{
    \begin{figure}[#1]
        \centering
        \includegraphics[width=#5\columnwidth]{#2}
        \caption[#3]{\textbf{#3}#4}
        \label{fig:#2}
    \end{figure}
}

\lstset{
frame=single,
breaklines=true,
columns=fullflexible
}


\title{\mytitle}
  \author{\myauthor\hspace{1em}\\\contact\\FWC220099    IITH-Future Wireless Communications     Assignment-1\hspace{0.5em}\hspace{0.5em}\mymodule}
\date{}
\hypersetup{pdfauthor=\myauthor,pdftitle=\mytitle,pdfkeywords=\mykeywords}
\sloppy
% #######################################
% ########### START FROM HERE ###########
% #######################################
 
 \begin{document}
 \maketitle
 \tableofcontents
    %\begin{figure}
        %\centering
        %\includegraphics[width=\linewidth]{}
        %\includegraphics[width=\linewidth]{}
    %    \caption{\textbf{Karnaugh Map}}
     %   \label{fig:my_label}
    %\end{figure}
  \textbf{}{\mykeywords}
\vspace{5mm}      
\section{Abstract}
This manual shows Realisation of boolean \\
expression from the given k-map by using arduino with AVR ASSEMBLY

\vspace{5mm}    
\section{Components}
     
       \begin{tabularx}{0.43\textwidth}{
  | >{\centering\arraybackslash}X
  | >{\centering\arraybackslash}X
  | >{\centering\arraybackslash}X | }
\hline
\textbf{Component}&\textbf{Value}& \textbf{Quantity}\\ \hline
Arduino   & UNO & 1 \\ \hline
Bread board   & - & 1 \\ \hline
Jumper wires  & M-M & 8 \\ \hline
Led           & - & 1\\ \hline
Resistor      & 150ohms & 1\\ \hline
\end{tabularx}
\begin{center}
   
\end{center}

\section{BooleanEquation}
By using boolean equation we write our code in assembly  we get the boolean equation as follows \\        Y = QR'+S

\section{Hardware Connections }
1.in arduino we are having pins P,Q,R,S.here we are using port B pin 8 is taken as output pin. \\
\hfill \break
2.port D pins 2,3,4,5 pins are taken as a inputs. portD pins 2,3,4,5 pins are connected vcc or gnd in breadboard as per truth table

\vspace{5mm}


\section{Procedure}

\textbf 1) Connect 5v of the Arduino to the top red of the bread board ang GND to the bottom green
\hfill \break
\hfill \break
\textbf 2) Connect b0 pin in the arduino to connect to one LED+
\hfill \break
\hfill \break
\textbf 3) Connect arduino d2 pin to the gnd or vcc according to inputs
\hfill \break
\hfill \break
\textbf 4) Connect arduino d3 pin to the gnd or vcc according to inputs
\hfill \break
\hfill \break
\textbf 5) Connect arduino d4 pin to the gnd or vcc according to inputs
\hfill \break
\hfill \break
\textbf 6) Connect arduino d5 pin to the gnd or vcc according to inputs
\hfill \break
\hfill \break
\textbf 6) Connect one LED+ to one end of the resisitor and other end of resistor to vcc and gnd the other terminal of LED
\hfill \break
\hfill \break
\textbf 7) Change the d2 d3 d4 d5 pins in the arduino from vcc to gnd as per truthtable and observe the outputs
\hfill \break

\vspace{5mm}

\section{Truth table for given K-map}
\begin{tabularx}{0.46\textwidth} { 
  | >{\centering\arraybackslash}X 
  | >{\centering\arraybackslash}X 
  | >{\centering\arraybackslash}X
  | >{\centering\arraybackslash}X 
  | >{\centering\arraybackslash}X | }
  \hline
 P & Q & R & S  & Y\\
\hline
0 & 0 & 0 & 0 & 0 \\  
\hline
0 & 0 & 0 & 1 & 1 \\ 
\hline
0 & 0 & 1 & 0 & 0 \\
\hline
0 & 0 & 1 & 1 & 1 \\
\hline
0 & 1 & 0 & 0 & 1 \\  
\hline
0 & 1 & 0 & 1 & 1 \\ 
\hline
0 & 1 & 1 & 0 & 0 \\
\hline
0 & 1 & 1 & 1 & 1 \\
\hline
1 & 0 & 0 & 0 & 0 \\
\hline
1 & 0 & 0 & 1 & 1 \\
\hline
1 & 0 & 1 & 0 & 0 \\
\hline
1 & 0 & 1 & 1 & 1 \\
\hline
1 & 1 & 0 & 0 & 1 \\
\hline
1 & 1 & 0 & 1 & 1 \\
\hline
1 & 1 & 1 & 0 & 0 \\
\hline
1 & 1 & 1 & 1 & 1 \\
\hline
\end{tabularx}
\begin{center}
TABLE 1
\end{center}


\section{Software}
\textbf{Execute the following code using the below provided link.}\\
\begin{center}
\fbox{\parbox{8.5cm}{\url{https://github.com/likhith1101/fwcassgn}}}
\end{center}
 
%\begin{document}

\vspace{5mm}  


%\end{document}

\end{document}

