\documentclass[12pt]{article}
\usepackage{graphicx}
\usepackage{amsmath}
\usepackage{mathtools}
\usepackage{gensymb}
\usepackage{amssymb}
\usepackage{tikz}
\usetikzlibrary{arrows,shapes,automata,petri,positioning,calc}
\usepackage{hyperref}
\usepackage{tikz}
\usetikzlibrary{matrix,calc}
\usepackage[margin=0.5in]{geometry}

\providecommand{\norm}[1]{\left\lVert#1\right\rVert}
\newcommand{\myvec}[1]{\ensuremath{\begin{pmatrix}#1\end{pmatrix}}}
\let\vec\mathbf
%\providecommand $${\norm}[1]{\left\lVert#1\right\rVert}$$
\providecommand{\abs}[1]{\left\vert#1\right\vert}
\let\vec\mathbf

\newcommand{\mydet}[1]{\ensuremath{\begin{vmatrix}#1\end{vmatrix}}}
\providecommand{\brak}[1]{\ensuremath{\left(#1\right)}}
\providecommand{\lbrak}[1]{\ensuremath{\left(#1\right.}}
\providecommand{\rbrak}[1]{\ensuremath{\left.#1\right)}}
\providecommand{\sbrak}[1]{\ensuremath{{}\left[#1\right]}}

\providecommand{\brak}[1]{\ensuremath{\left(#1\right)}}
\providecommand{\norm}[1]{\left\lVert#1\right\rVert}
\newcommand{\solution}{\noindent \textbf{Solution: }}

\let\vec\mathbf
\def\inputGnumericTable{}
\usepackage{color}                                            %%
    \usepackage{array}                                            %%
    \usepackage{longtable}                                        %%
    \usepackage{calc}                                             %%
    \usepackage{multirow}                                         %%
    \usepackage{hhline}                                           %%
    \usepackage{ifthen}
\usepackage{array}
\usepackage{amsmath}   % for having text in math mode
\usepackage{listings}
\lstset{
language=tex,
frame=single, 
breaklines=true
}
\newenvironment{Figure}
  {\par\medskip\noindent\minipage{\linewidth}}
  {\endminipage\par\medskip}
\begin{document}
\begin{center}
\textbf\large{CLASS-9\\CHAPTER-10 \\ CIRCLES}

\end{center}
\section*{Excercise 10.6}

Q1. Prove that the line of centres of two intersecting circles subtends equal angles at the two points of intersection.
\section*{\large Solution}:
\begin{figure}[h!]
\centering
\includegraphics[width=\columnwidth]{figs/circle3.png}
\caption{}
\label{fig:Fig1}
\end{figure}


\section*{\large Construction}:

\begin{table}[h!]
	\small
	\centering
	%\subimport{../tables/}{table1.tex}
     \input{tables/table1.tex}
%	\caption{}
	\label{table:table1}
\end{table}


\section*{\large Verification:}

 The two circle equations are given by:
\begin{align}
\label{eq:1}
	\norm{x}^2-9=0\\
	\norm{x}^2-8\vec{e}_1+12=0
\end{align}
Equation of two conics is given by:
 \begin{align}
 \vec{x}^\top\vec{V}_i\vec{x}+2\vec{u}_i^\top\vec{x}+f_i=0, \quad i=1,2
 \label{eq:3}
 \end{align}
 Represent the two circles in conic form:
 \begin{align}
	\vec{x}^\top\vec{x}-9=0\\
	\vec{x}^\top\vec{x}+2\myvec{-4&0}+12=0
\end{align}
On comparing above two equations with \eqref{eq:3}, we get:
 \begin{align}
	  \vec{V}_1&=\vec{I},\vec{u}_1=\myvec{0\\0},f_1=-9\\
	  \vec{V}_2&=\vec{I},\vec{u}_2=\myvec{-4\\0},f_2=12
\end{align}
The intersection of the given conics is obtained
as
\begin{align}
	\label{eq:8}
\vec{V}_1+\mu\vec{V}_2&= \myvec{
\mu+1 & 0\\
0 & \mu+1
}
\\ \label{eq:9}
\vec{u}_1+\mu\vec{u}_2&= \myvec{
4\\
0
}
\\ \label{eq:10}
f_1+\mu f_2&= -21
\end{align}
This conic is a single straight line if and only if, 
\begin{align}
\mydet{\vec{V}_1 + \mu\vec{V}_2 & \vec{u}_1+\mu \vec{u}_2\\ (\vec{u}_1+\mu \vec{u}_2)^{\top} & f_1 + \mu f_2} &= 0
\label{eq:11}
\end{align}
Substituting equation \eqref{eq:8},\eqref{eq:9} and \eqref{eq:10} in equation \eqref{eq:11}:
\begin{align}
\implies \mydet{1+\mu& 0 & -4\mu\\ 
0 & 1+\mu & 0 \\
-4\mu & 0 & -9+12\mu
} &= 0
\end{align}
Solving the above equation we get,
\begin{align}
    \mu = -1
\end{align}
Thus, the parameters for a straight line can be expressed as
 \begin{align}
 \label{eq:14}
	\vec{V} &= 
\vec{V}_1 + \mu\vec{V}_2
=\myvec{ 0 & 0 \\ 0 & 0},
\\ \label{eq:15}
	\vec{u} &=
\vec{u}_1+\mu \vec{u}_2
	= \myvec{
4\\
0
},
\\ \label{eq:16}
f&=f_1 + \mu f_2=-21
\end{align}
The conic equation is given by:
 \begin{align}
 \vec{x}^\top\vec{V}\vec{x}+2\vec{u}^\top\vec{x}+f=0, 
  \label{eq:17}
 \end{align}
By substituting \eqref{eq:14},\eqref{eq:15} and \eqref{eq:16} in conic equation \eqref{eq:17}, we get point of contact $\vec{q}$::
\begin{align}
	\myvec{x&y}\myvec{0&0\\0&0}\myvec{x\\y}+2\myvec{4&0}\myvec{x\\y}-21&=0\\
\implies	8x&=21\\
\implies	x&=\frac{21}{8}
\end{align}
	The point of contact $\vec{q}$ is given by:
	\begin{align}
	\vec{q}&=x\vec{e}_1=\myvec{\frac{21}{8}\\[2pt]0}
\end{align}
The points of intersection of line is given by: 
\begin{align}
L: \quad \vec{x} = \vec{q} + \kappa \vec{m} \quad \kappa \in \mathbb{R}
\end{align}
with the conic section, we have:
 \begin{align}
 \vec{x}_i = \vec{q} + \kappa_i \vec{m}
 \end{align}
 where, 
\begin{align}
\kappa_i = \frac{1}
{\vec{m}^T\vec{V}\vec{m}}
\lbrak{-\vec{m}^T\brak{\vec{V}\vec{q}+\vec{u}}}
\pm
\rbrak{\sqrt{
\sbrak{
\vec{m}^T\brak{\vec{V}\vec{q}+\vec{u}}
}^2
-
\brak
{
\vec{q}^T\vec{V}\vec{q} + 2\vec{u}^T\vec{q} +f
}
\brak{\vec{m}^T\vec{V}\vec{m}}
}
}
\label{eq:24}
\end{align}
On substituting the below values in \eqref{eq:24}
\begin{align}
 \vec{m}=\vec{e}_2,\vec{V}=\vec{I},\vec{u}=\myvec{0\\0},\vec{q}=\myvec{\frac{13}{5}\\[2pt]0},f=-9
\end{align}
We get,
\begin{align}
\kappa_i=-\frac{29}{20},+\frac{29}{20}
\end{align}
The intersecting points $\vec{C}$ and $\vec{D}$ are given by:
\begin{align}
    \vec{C}&=\vec{q}+\kappa_1\vec{m}=\myvec{\frac{21}{8}\\[2pt]-\frac{29}{20}}\\
    \vec{D}&=\vec{q}+\kappa_2\vec{m}=\myvec{\frac{21}{8}\\[2pt]\frac{29}{20}}
\end{align}
Check whether the intersection angles $\angle$ADB and $\angle$ACB are equal or not:
\begin{enumerate}
\item Finding $\angle$ADB:
	\begin{align}
		 \vec{A-D} = \myvec{-\frac{21}{8}\\[2pt]-\frac{29}{20}},
		\vec{B-D}& = \myvec{\frac{11}{8}\\[2pt]-\frac{21}{8}}\\
	 \vec{(A-D)^\top(B-D)}&= -\frac{3}{2}\\
	 \norm{\vec{A-D}}\norm{\vec{C-D}}& = 6\\
		\cos(\angle ADB)& = \frac{\vec{(A-D)^\top(B-D)}}{\norm{\vec{A-D}}\norm{\vec{B-D}}}\\
		\angle ADB&=104\degree
\end{align}
\item Finding $\angle$ACB:
\begin{align}
	\vec{A-C} = \myvec{-\frac{21}{8}\\[2pt]\frac{29}{20}},
	 \vec{B-C}& = \myvec{\frac{11}{8}\\[2pt]\frac{29}{20}}\\
	 \vec{(A-C)^\top(B-C)}&= -\frac{3}{2}\\
	 \norm{\vec{A-C}}\norm{\vec{B-C}}& = 6\\
	 \cos(\angle ACB) &= \frac{\vec{(A-C)^\top(B-C)}}{\norm{\vec{A-C}}\norm{\vec{B-C}}}\\
	 \angle ACB&=104\degree
\end{align}
\end{enumerate}
Hence, both the intersecting angles are equal to each other, which satisfies the above condition.


\end{document}
