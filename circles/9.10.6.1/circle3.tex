\documentclass[12pt]{article}
\usepackage{graphicx}
\usepackage{amsmath}
\usepackage{mathtools}
\usepackage{gensymb}

\newcommand{\mydet}[1]{\ensuremath{\begin{vmatrix}#1\end{vmatrix}}}
\providecommand{\brak}[1]{\ensuremath{\left(#1\right)}}
\providecommand{\norm}[1]{\left\lVert#1\right\rVert}
\newcommand{\solution}{\noindent \textbf{Solution: }}
\newcommand{\myvec}[1]{\ensuremath{\begin{pmatrix}#1\end{pmatrix}}}
\let\vec\mathbf
\def\inputGnumericTable{}
\usepackage{color}                                            %%
    \usepackage{array}                                            %%
    \usepackage{longtable}                                        %%
    \usepackage{calc}                                             %%
    \usepackage{multirow}                                         %%
    \usepackage{hhline}                                           %%
    \usepackage{ifthen}
\usepackage{array}
\usepackage{amsmath}   % for having text in math mode
\usepackage{listings}
\lstset{
language=tex,
frame=single, 
breaklines=true
}
\begin{document}
\begin{center}
\textbf\large{CLASS-9\\CHAPTER-10 \\ CIRCLES}

\end{center}
\section*{Excercise 10.6}

Q1. Prove that the line of centres of two intersecting circles subtends equal angles at the two points of intersection.
\section*{\large Solution}
\begin{figure}[h!]
\centering
\includegraphics[width=\columnwidth]{figs/circle2.png}
\caption{}
\label{fig:Fig1}
\end{figure}
\section*{\large Construction}

\begin{table}[h!]
	\small
	\centering
	%\subimport{../tables/}{table1.tex}
     \input{tables/table1.tex}
%	\caption{}
	\label{table:table1}
	\end{table}



\section*{Verification:}
 The two circle equations given by:
\begin{align}
	\norm{x}^2-9=0\\
	\norm{x}^2-8\vec{e}_1+12=0
\end{align}
 It is easy to verify that
\begin{align}
	\vec{q}=2.62\vec{e_1}
\end{align}
By substituting the below values, we get intersecting points: 
 \begin{align}
	 \vec{m}=\vec{e}_2,\vec{q}=2.62\vec{e}_1,\vec{V}=\vec{I},\vec{u}=\vec{0},f=-9
\end{align}
The intersecing points $\vec{C}$ and $\vec{D}$
\begin{align}
    \vec{C}=\myvec{
2.62\\
-1.46
    },
    \vec{D}=\myvec{
2.62\\
1.46
    }
\end{align}
\\
Check whether the intersection angles $\angle$ADB and $\angle$ACB are equal or not:
\begin{enumerate}
\item Finding $\angle$ADB:
	\begin{align}
		 \vec{A-D} = \myvec{-2.62\\-1.46},
		\vec{B-D}& = \myvec{1.3\\-1.46}\\
	 \vec{(A-D)^\top(B-D)}&= -0.22\\
	 \norm{\vec{A-D}}\norm{\vec{C-D}}& = 5.8\\
		\cos(\angle ADB)& = \frac{\vec{(A-D)^\top(B-D)}}{\norm{\vec{A-D}}\norm{\vec{B-D}}}\\
		\angle ADB&=104.2\degree
\end{align}
\item Finding $\angle$ACB:
\begin{align}
	\vec{A-C} = \myvec{-2.62\\1.46},
	 \vec{B-C}& = \myvec{1.3\\1.46}\\
	 \vec{(A-C)^\top(B-C)}&= -0.22\\
	 \norm{\vec{A-C}}\norm{\vec{B-C}}& = 5.8\\
	 \cos(\angle ACB) &= \frac{\vec{(A-C)^\top(B-C)}}{\norm{\vec{A-C}}\norm{\vec{B-C}}}\\
	 \angle ACB&=104.2\degree
\end{align}
\end{enumerate}
Hence, both the intersecting angles are equal to each other, which satisfies the above condition.


\end{document}
